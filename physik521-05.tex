% Für Seitenformatierung

\documentclass[DIV=15]{scrartcl}

% Zeilenumbrüche

\parindent 0pt
\parskip 6pt

% Für deutsche Buchstaben und Synthax

\usepackage[ngerman]{babel}

% Für Auflistung mit speziellen Aufzählungszeichen

\usepackage{paralist}

% zB für \del, \dif und andere Mathebefehle

\usepackage{amsmath}
\usepackage{commath}
\usepackage{amssymb}

% für nicht kursive griechische Buchstaben

\usepackage{txfonts}

% Für \SIunit[]{} und \num in deutschem Stil

\usepackage[output-decimal-marker={,}]{siunitx}
\usepackage[utf8]{inputenc}

% Für \sfrac{}{}, also inline-frac

\usepackage{xfrac}

% Für Einbinden von pdf-Grafiken

\usepackage{graphicx}

% Umfließen von Bildern

\usepackage{floatflt}

% Für Links nach außen und innerhalb des Dokumentes

\usepackage{hyperref}

% Für weitere Farben

\usepackage{color}

% Für Streichen von z.B. $\rightarrow$

\usepackage{centernot}

% Für Befehl \cancel{}

\usepackage{cancel}

% Für Layout von Links

\hypersetup{
	citecolor=black,
	colorlinks=true,
	linkcolor=black,
	urlcolor=blue,
}

% Verschiedene Mathematik-Hilfen

\newcommand \e[1]{\cdot10^{#1}}
\newcommand\p{\partial}

\newcommand\half{\frac 12}
\newcommand\shalf{\sfrac12}

\newcommand\skp[2]{\left\langle#1,#2\right\rangle}
\newcommand\mw[1]{\left\langle#1\right\rangle}

\newcommand \eexp[1]{\exp\del{#1}}

% Nabla und Kombinationen von Nabla

\renewcommand\div[1]{\skp{\nabla}{#1}}
\newcommand\rot{\nabla\times}
\newcommand\grad[1]{\nabla#1}
\newcommand\laplace{\triangle}
\newcommand\dalambert{\mathop{{}\Box}\nolimits}

%Für komplexe Zahlen

\newcommand \ii{\mathrm i}
\renewcommand{\Im}{\mathop{{}\mathrm{Im}}\nolimits}
\renewcommand{\Re}{\mathop{{}\mathrm{Re}}\nolimits}

%Für Bra-Ket-Notation

\newcommand\bra[1]{\left\langle#1\right|}
\newcommand\ket[1]{\left|#1\right\rangle}
\newcommand\braket[2]{\left\langle#1\left.\vphantom{#1 #2}\right|#2\right\rangle}
\newcommand\braopket[3]{\left\langle#1\left.\vphantom{#1 #2 #3}\right|#2\left.\vphantom{#1 #2 #3}\right|#3\right\rangle}


\setcounter{section}{0}
\renewcommand\thesection{H\,5.\arabic{section}}
\renewcommand\thesubsection{\thesection.\alph{subsection}}

\title{physik521: Übungsblatt 05}
\author{%
    Lino Lemmer \\ \small{\texttt{s6lilemm@uni-bonn.de}}
    \and
    Martin Ueding \\ \small{\texttt{mu@martin-ueding.de}}
    \and
    Paul Manz \\ \small{\texttt{p.m@uni-bonn.de}}
}

\begin{document}
\maketitle
\section{Sattelpunktmethode}
\subsection{}
Gegeben ist das Integral
\begin{align}
    I &= \lim_{N\to\infty}\int_a^b\!\dif x\; \exp{Nf(x)} \label{eq:integral}
    \intertext{$f$ hat ein globales Maximum bei $x=x_0$. Um diesen Punkt
    Taylorn wir die Funktion:}
    f(x) &= f\del{x_0} + f'\del{x_0}\del{x-x_0} +
    \frac{f''\del{x_0}}2\del{x-x_0}^2 + \dots
    \label{eq:taylor}
    \intertext{%
        Setzen wir \eqref{eq:taylor} in \eqref{eq:integral} ein und
        brechen nach dem dritten Glied ab, erhalten wir
    }
    I &= \lim_{N\to\infty}\int_a^b\!\dif x\;
    \exp{Nf\del{x_0} + Nf'\del{x_0}x + \frac{N}2f''
    \del{x_0}\del{x-x_0}^2}\notag
    \intertext{%
        Die erste Ableitung verschwindet bei $x_0$, da dort das Maximum der
        Funktion ist.
    }
    &= \lim_{N\to\infty}\int_a^b\!\dif x\;
    \exp{Nf\del{x_0}}\exp{\frac{N}2f''\del{x_0}\del{x-x_0}^2}\notag \\
    &= \lim_{N\to\infty}\exp{Nf\del{x_0}}\int_a^b\!\dif x\;
    \exp{\frac{N}2f''\del{x_0}\del{x-x_0}^2}\notag \\
    &= \lim_{N\to\infty}\exp{Nf\del{x_0}}\int_a^b\!\dif x\;
    \exp{\frac N2f''\del{x_0}x^2}\exp{Nf''\del{x_0}x_0x}
    \exp{\frac N2f''\del{x_0}x_0^2} \notag \\
    &= \lim_{N\to\infty}\exp{Nf\del{x_0}}\exp{\frac N2f''\del{x_0}x_0^2}
    \int_a^b\!\dif x\;
    \exp{\frac N2f''\del{x_0}x^2}\exp{Nf''\del{x_0}x_0x} \notag
    \intertext{mit $a = -\frac N2 f''\del{x_0}$:}
    &= \lim_{N\to\infty} \exp{Nf\del{x_0}} \exp{-ax_0^2} \int_a^b\!\dif x\;
    \exp{2ax_0x}\exp{-ax^2} \notag
    \intertext{mit $\ii\omega = -2ax_0$}
    &= \lim_{N\to\infty} \exp{Nf\del{x_0}} \exp{\frac{\omega^2}{4a}} \int_a^b
    \!\dif x\; \exp{-\ii\omega x}\exp{-ax^2} \label{eq:integral_getaylort}
    \intertext{%
        Nun betrachten wir die Formel für Gaußsche Integrale
    }
    \sqrt{\frac{\piup}a}\exp{-\frac{\omega^2}{4a}}&=
    \int\!\dif t\;\exp{-\ii\omega t}\exp{-at^2}\label{eq:Gauss} 
    \intertext{%
        Lösen wir nun \eqref{eq:integral_getaylort} mit \eqref{eq:Gauss}
        auf, erhalten wir:
    }
    &= \lim_{N\to\infty} \exp{Nf\del{x_0}} \exp{\frac{\omega^2}{4a}}
    \sqrt{\frac\piup a}\exp{-\frac{\omega^2}{4a}}\notag \\
    &= \lim_{N\to\infty} \exp{Nf\del{x_0}} \sqrt{\frac\piup a}\notag \\
    &= \lim_{N\to\infty} \exp{Nf\del{x_0}}
    \sqrt{-\frac{2\piup}{Nf''\del{x_0}}}\notag
    \intertext{%
        Da die zweite Ableitung an der Maximalstelle negativ ist, folgt
    }
    &= \lim_{N\to\infty} \exp{Nf\del{x_0}}
    \sqrt{\frac{2\piup}{N\abs{f''\del{x_0}}}}\label{eq:sattelpunktmethode}
\end{align}
\subsection{}

\begin{align*}
    \lim_{N\to\infty} N! &= \lim_{N\to\infty} \Gamma\del{N+1} \\
                         &= \lim_{N\to\infty} \int_0^\infty \!\dif x \;
    x^N\exp{-x} \\
    \intertext{%
        Wir substituieren nun $x = Nz$ und $N \dif z = \dif x$:
    }
    &=\lim_{N\to\infty} \int_0^\infty \!\dif z \; NN^Nz^N \exp{-Nz} \\
    &=\lim_{N\to\infty} NN^N \int_0^\infty \!\dif z \; 
    \exp{N\log\del{z}} \exp{-Nz} \\
    &=\lim_{N\to\infty} N^{N+1} \int_0^\infty \!\dif z \;
    \exp{N\del{\log\del{z}-z}}
    \intertext{%
        Wir setzen nun $f(z) = \log\del z - z$. Diese Funktion hat ihr
        Maximum bei $z = 1$. Die zweite Ableitung an dieser Stelle ist
        $f''(1) = -\frac{1}{1^2} = -1$. Nach \eqref{eq:sattelpunktmethode}
        gilt
    }
    &= \lim_{N\to\infty} N^{N+1}\exp{-N}\sqrt{\frac{2\piup}{N}}\\
    &=\lim_{N\to\infty}\sqrt{2\piup N}N^N\exp{-N}
\end{align*}
Dies ist die gesuchte Stirling-Formel.

\section{Ensemble quantenmechanischer harmonischer Oszillatoren}
\subsection{}
\subsection{}
\subsection{}
\subsection{}
\subsection{}
\end{document}
